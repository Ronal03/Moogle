
\section{Clase Coseno}
\begin{frame}
    \frametitle{Coseno}
    \textbf{Metodos}

    \begin{itemize}
        \item \underline{Similitud} que devuelve un valor de tipo \textit{double} que representa la similitud
        de 2 vectores que recibe como parámetros aplicando la fórmula del
        producto punto y calculando las magnitudes de cada uno, mientras
        mayor sea el número más semejantes son los vectores.

        \item \underline{WordsInDoc} que recibe una lista de palabras a comparar con el
        contenido de un documento y devuelve una nueva lista con las palabras
        de la lista del parámetro que están en él.
    
        \item \underline{Positions} recibe esas palabras que coincidieron y la lista de todas las
        palabras en total y devuelve una lista nueva con las posiciones de esas
        palabras en la lista de todas las palabras.
    \end{itemize}
\end{frame}
    
\begin{frame}
    \frametitle{Coseno}
    \begin{itemize}
        \item \underline{TFIDFWordsQueryInDoc} que recibe la matriz con los TFIDF
        la lista
        de las posiciones y el número del documento que contenía a esas
        palabras y devuelve una lista de los TFIDF de esas palabras en ese
        documento
        \item \underline{Snippet} que recibe a la palabra con mayor TFIDF y al documento con
        el contenido original, el método crea una lista separando las cadenas
        del texto por un espacio, y aplico el método \underline{LevenshteinDistance} de la
        clase \textbf{Suggestion} entre cada porción del texto y esa palabra, la distancia
        menor será la porción más parecida a la palabra, a esa porción le sumo
        una cantidad a la derecha y a la izquierda y esa cadena la retorna como
        el valor del snippet.
    \end{itemize}
       
\end{frame}
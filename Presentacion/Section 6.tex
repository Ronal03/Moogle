
\section{Funcionamiento}
\begin{frame}
    \frametitle{Funcionamiento}
    En el método \underline{Query} de la clase \textbf{Moogle} lo primero que hice es llevar el
query a una lista en minúsculas y quitarle todos los caracteres que no
coincidan con letras o números. Después busco cada documento de la
carpeta \emph{Content} y voy creando las listas con el contenido de cada
documento, una con el contenido original y otra con el contenido
modificado después de quitarles los caracteres que no sean números y
letras, además de la lista de los títulos sin extensiones. Si alguno de los
documentos esta vacío se elimina ese documento de cada lista. Después
creo la lista de todas las palabras q hay y elimino las que sean iguales y
creo las matrices de los TFIDF de cada una .Creo el diccionario \emph{Vocabulary}
con todas las palabras y se lo paso como parámetro al método
\underline{WordsInVocabulary} con las palabras del query. Tomo los valores del
diccionario y creo una lista con ellos, y voy aplicando el método \underline{Similitud}
con cada columna de la matriz TFIDF y cada valor se lo agrego a la lista
de \emph{Scores}.
\end{frame}
\begin{frame}
    \frametitle{Funcionamiento}
    i todos los valores de la lista son 0, es porque todas las
palabras aparecen en todos los documentos y en ese caso cálculo la
similitud con cada columna de la matriz TF para quedarme con el
documento de mayor TF. Después voy agregando cada \emph{Score} distinto de
0 con el título correspondiente a un diccionario \emph{ScoresTitle} y eso mismo
hago con el diccionario \emph{ScoreDocs} que contiene como clave los números
de orden de los documentos. Le aplico el método \underline{SortDictionary} a cada
uno y la lista de los títulos y los scores las modifico con la lista de los títulos
y scores de los diccionarios en orden invertido. Creo una lista invertida
\emph{DocNumbers} también con los números de los documentos. Con un ciclo
\emph{for} recorro todos los documentos que coincidan con la lista \emph{DocNumbers} y
en cada uno busco la palabra de mayor TFIDF, calculo el
snippet y lo agrego a una lista de \emph{Snippet}. Creo un \emph{array }item de tipo
\emph{SearchItem} con cada uno de los títulos,snippets y scores
correspondientes.
\end{frame}
\begin{frame}
    \frametitle{Funcionamiento}
    Para buscar la sugerencia creo un \emph{array} con el query y
un \emph{array} tester de igual longitud para recorrer cada documento; declaro una
variable de apoyo con valor 5.Llevo cada documento a un \emph{array} y con el
tester convertido a \emph{string} texte voy a compararlo con el query por el método
\underline{LevenshteinDistance}, me quedo con el \emph{string} mas similar al query, y si no
encuentra una similitud $<=$ 5,la sugerencia será una cadena vacía. Por último le paso el \emph{array} ítem y la sugerencia como parámetros al objeto de
tipo \emph{SearchResult} que será devuelto.
\end{frame}


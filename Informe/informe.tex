\documentclass[a4paper,12pt]{article}
\usepackage{titlesec}
\usepackage{graphicx}
\titleformat{\section}[display]{\bfseries\Large\itshape}{Section No. \thesection}
{0.5ex}
{ \vspace{1ex}
\centering}[\vspace{1ex}]


\title{Proyecto Moogle}
\author{Ronal Prieto Vazques   C-112}
\date{\today}
\begin{document}
\begin{figure}[t]
    \hspace{13cm}
\includegraphics[width=2cm,height=2cm]{Logo de Matcom.png}
\end{figure}
\maketitle

\tableofcontents
\begin{figure}[t]
    
        \hspace{13cm}
    \includegraphics[width=2cm,height=2cm]{Logo de Matcom.png}
\end{figure}
\newpage
\section{Clase Dict}
\large\textbf{Metodos}

\begin{itemize}
    \item \underline{ConvertToDict}: acepta como parámetro una lista de \textit{strings }y devuelve
    un diccionario donde la clave de cada elemento es una palabra de la
    lista y cada valor lo inicio con 0.
    \item Dos metodos \underline{SortDictionary} uno con claves de tipo \textit{string} y otro con
    claves de tipo \textit{int}, cada método ordena los elementos del diccionario
    según los valores que son de tipo \textit{double}.
    \item \underline{WordsInVocabulary} que acepta una lista de \textit{strings} y un diccionario, no
    devuelve nada, sino modifica el diccionario y las palabras que coincidan
    con las de la lista toman valores de 1.
\end{itemize}
\section{Clase Coseno}
\large\textbf{Metodos}
\begin{itemize}
    \item \underline{Similitud} que devuelve un valor de tipo \textit{double} que representa la similitud
    de 2 vectores que recibe como parámetros aplicando la fórmula del
    producto punto y calculando las magnitudes de cada uno, mientras
    mayor sea el número más semejantes son los vectores.
    \begin{figure}[t]
        \hspace{13cm}
    \includegraphics[width=2cm,height=2cm]{Logo de Matcom.png}
    \end{figure}
    \item \underline{WordsInDoc} que recibe una lista de palabras a comparar con el
    contenido de un documento y devuelve una nueva lista con las palabras
    de la lista del parámetro que están en él.
    \item \underline{Positions} recibe esas palabras que coincidieron y la lista de todas las
    palabras en total y devuelve una lista nueva con las posiciones de esas
    palabras en la lista de todas las palabras.
    \item \underline{TFIDFWordsQueryInDoc} que recibe la matriz con los TFIDF
    la lista
    de las posiciones y el número del documento que contenía a esas
    palabras y devuelve una lista de los TFIDF de esas palabras en ese
    documento
    \item \underline{Snippet} que recibe a la palabra con mayor TFIDF y al documento con
    el contenido original, el método crea una lista separando las cadenas
    del texto por un espacio, y aplico el método \underline{LevenshteinDistance} de la
    clase \textbf{Suggestion} entre cada porción del texto y esa palabra, la distancia
    menor será la porción más parecida a la palabra, a esa porción le sumo
    una cantidad a la derecha y a la izquierda y esa cadena la retorna como
    el valor del snippet.
\end{itemize}
\newpage
\begin{figure}[t]
    \hspace{13cm}
\includegraphics[width=2cm,height=2cm]{Logo de Matcom.png}
\end{figure}
\section{Clase TFIDF}
\large\textbf{Metodos}
\begin{itemize}
  \item \underline{TF} que devuelve una matriz ,con los TF de todas las palabras en cada
  documento
  \item \underline{TFIDF} que devuelve una matriz con los valores TFIDF de todas las
  palabras en todos los documentos.
\end{itemize}
\section{Clase Suggestion}
\large\textbf{Metodos}
\begin{itemize}
    \item \underline{LevenshteinDistance} explicado anteriormente
    \item \underline{ConvertDocArray} que me lleva el contenido de un documento a un
    \textit{arrray } de \textit{string}.
\end{itemize}
\section{Clase Matrix}
En el constructor acepta una matriz de tipo \textit{double} y tiene una
propiedad \textbf{Row} con la parte \textit{set} privada que devuelve la matriz.Contiene
9 métodos cada uno realiza la operación algebraica que indica el nombre. \\ \\
\large\textbf{Metodos}
\begin{itemize}
    \item \underline{Suma} suma 2 matrices
    \item \underline{Producto} multiplica 2 matrices
    \item \underline{Traspuesta} calcula la traspuesta de una matriz
    \item \underline{Deter} Calcula el determinante de orden n de una matriz
    \item \underline{MatrizXVector} multiplica una matriz por un vector
    \item \underline{MatrizXEscalar} multiplica una matriz por un escalar
    \item \underline{VectorSuma} suma 2 vectores
    \item \underline{VectorMult} multiplica 2 vectores
    \item \underline{VectorXescalar} multiplica un vector por un escalar
\end{itemize}
\begin{figure}[t]
    \hspace{13cm}
\includegraphics[width=2cm,height=2cm]{Logo de Matcom.png}
\end{figure}
\section{Funcionamiento}
En el método \underline{Query} de la clase \textbf{Moogle} lo primero que hice es llevar el
query a una lista en minúsculas y quitarle todos los caracteres que no
coincidan con letras o números. Después busco cada documento de la
carpeta \emph{Content} y voy creando las listas con el contenido de cada
documento, una con el contenido original y otra con el contenido
modificado después de quitarles los caracteres que no sean números y
letras, además de la lista de los títulos sin extensiones. Si alguno de los
documentos esta vacío se elimina ese documento de cada lista. Después
creo la lista de todas las palabras q hay y elimino las que sean iguales y
creo las matrices de los TFIDF de cada una .Creo el diccionario \emph{Vocabulary}
con todas las palabras y se lo paso como parámetro al método
\underline{WordsInVocabulary} con las palabras del query. Tomo los valores del
diccionario y creo una lista con ellos, y voy aplicando el método \underline{Similitud}
con cada columna de la matriz TFIDF y cada valor se lo agrego a la lista
de \emph{Scores}. Si todos los valores de la lista son 0, es porque todas las
palabras aparecen en todos los documentos y en ese caso cálculo la
similitud con cada columna de la matriz TF para quedarme con el
documento de mayor TF. Después voy agregando cada \emph{Score} distinto de
0 con el título correspondiente a un diccionario \emph{ScoresTitle} y eso mismo
hago con el diccionario \emph{ScoreDocs} que contiene como clave los números
de orden de los documentos. Le aplico el método \underline{SortDictionary} a cada
uno y la lista de los títulos y los scores las modifico con la lista de los títulos
y scores de los diccionarios en orden invertido. Creo una lista invertida
\emph{DocNumbers} también con los números de los documentos. Con un ciclo
\emph{for} recorro todos los documentos que coincidan con la lista \emph{DocNumbers} y
en cada uno busco la palabra de mayor TFIDF, calculo el
snippet y lo agrego a una lista de \emph{Snippet}. Creo un \emph{array }item de tipo
\emph{SearchItem} con cada uno de los títulos,snippets y scores
correspondientes.\\
\begin{figure}[t]
    \hspace{13cm}
\includegraphics[width=2cm,height=2cm]{Logo de Matcom.png}
\end{figure}
Para buscar la sugerencia creo un \emph{array} con el query y
un \emph{array} tester de igual longitud para recorrer cada documento; declaro una
variable de apoyo con valor 5.Llevo cada documento a un \emph{array} y con el
tester convertido a \emph{string} texte voy a compararlo con el query por el método
\underline{LevenshteinDistance}, me quedo con el \emph{string} mas similar al query, y si no
encuentra una similitud $<=$ 5,la sugerencia será una cadena vacía. Por último le paso el \emph{array} ítem y la sugerencia como parámetros al objeto de
tipo \emph{SearchResult} que será devuelto.
\begin{figure}[b]
    \hspace{13cm}
    \includegraphics[width=2cm,height=2cm]{Logo de Matcom.png}
\end{figure}

\end{document}
